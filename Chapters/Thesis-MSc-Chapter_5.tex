% #############################################################################
% This is Chapter 5
% !TEX root = ../main.tex
% #############################################################################
% Change the Name of the Chapter i the following line
\fancychapter{Conclusions and Future Work}
\cleardoublepage
% The following line allows to ref this chapter
\label{chap:conclusion}

\section{Conclusions}

We began this study with the aim of being able to provide the \ac{EVCS} with more information regarding the power behavior of the building. With more information, this system should be able to make more informed decisions regarding the processing of electric car loads. We then introduced the challenge: "How can one forecast the power available in a near future, in order to optimize the \ac{EVCS}?". In this thesis, data from different sources (radiation, meteorology, production and consumption) was used in the construction of a predictive model whose outputs can used as inputs in the \ac{EVCS}. The result of the research is a set of models capable of producing a maximum and minimum power output forecast algorithm to boost the capabilities of the \ac{EVCS} by predicting \ac{EDP}'s building power availability for the next 5, 10 and 15 minutes. We proposed six different solutions to address this challenge, two classic vanilla solutions, in which simple typologies of \ac{RNN}s, \ac{GRU} and \ac{LSTM} are applied, two Encoder-Decoder models that use the same types of \ac{RNN}s in a more complex way in order to predict whole sequences, and two hybrid models that combine the predictive capabilities of Encoder-Decoder models with the extrapolative capabilities of \ac{CNN}s. Data from January 25, 2020 to June 12 of the same year was used in order to train, validate and test a total of 336 different combinations of hyperparameters, of which six were chosen as final models. Of the six, the one that presented the best results was .....xxxxxx

Some main contributions of this thesis are:
\begin{itemize}
\setlength\itemsep{0.1em}
    \item Automation of data manipulation treatment from various sources;
    \item Development of a \ac{ML} system capable of making forecasts for the available power in the building in 5, 10 and 15 minutes;
    \item Development, evaluation and comparison of the performance of six different predictive architectures;
    \item Availability of the code developed during the research.
\end{itemize}



\section{Future Work}

When it comes to future work, there are numerous factors that can be explored. As for the development of the models developed, it can still be explored: 

\begin{itemize}
\setlength\itemsep{0.1em}
    \item Deepen the role that each of the features has in the forecast results obtained;
    \item Building a system that uses, parallel to the variables registered in the past, future variables that are certain, such as the day of the week, month, holiday, etc.
    \item Development of a stacked system that combines the capabilities of the different models developed, and construction of a voting system capable of defining the best model to use for each specific situation.
    \item Development of models with a superior time goal;
    \item Use of a time window of data of several years, which allows the development of models capable of identifying patterns in the long term, which can be used in the enrichment of predictive models.
\end{itemize}