% #############################################################################
% RESUMO em Português
% !TEX root = ../main.tex
% #############################################################################
% use \noindent in firts paragraph
\noindent O edifício da \ac{EDP} localizado em Lisboa, Portugal está equipado com um sistema \ac{PV} responsável por gerar 70 kWp (kilowatt “pico”). Adicionalmente, é ainda composto por um sistema de carregamento de veículos elétricos \ac{EVCS}, capaz de variar a potência utilizada com base na potência global disponível no edifício. Embora o \ac{EVCS} tenha acesso à potência atual disponível no edifício, seria possível otimizar o sue funcionamento fornecendo-lhe também previsões da potência disponível no edifício num futuro próximo, por exemplo, com valores para daqui a 5, 10 e 15 minutos. Para dar resposta a esta necessidade, introduzimos um sistema composto por técnicas de Aprendizagem Automática. Foram utilizados diversos fatores, tanto meteorológicos como energéticos para prever a potência futura disponível. A implementação foi feita recorrendo a três arquiteturas diferentes, \ac{RNN} Simples, Codificador-Descodificador e \ac{1D CNN}-\ac{RNN}-Codificador-Descodificador. Para cada uma das três arquiteturas, foram testadas \ac{RNN}s do tipo \ac{GRU} e do tipo \ac{LSTM}, resultando num total de 6 modelos distintos: \ac{GRU} simples, \ac{LSTM} simples, \ac{GRU}-Codificador-Descodificador, \ac{LSTM}-Codificador-Descodificador, \ac{1D CNN}-\ac{GRU}-Codificador-Descodificador e \ac{1D CNN}-\ac{LSTM}-Codificador-Descodificador. Foi ainda testada a adição de Dropout de Monte Carlo às arquiteturas propostas, permitindo uma interpretação probabilística dos resultados obtidos, o que resultou em média, numa melhoria de xxx no desempenho dos modelos, e permitiu ainda o estabelecimento de intervalos de confiança para as previsões efetuadas. 