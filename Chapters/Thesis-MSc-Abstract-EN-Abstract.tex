% #############################################################################
% Abstract Text
% !TEX root = ../main.tex
% #############################################################################
% use \noindent in firts paragraph
\noindent The \ac{EDP} building located in Lisbon, Portugal, is equipped with a \ac{PV} system responsible for generating 70 kWp (kilowatt "peak"). In addition, it comprises an \ac{EVCS}, capable of adjusting the power used according to the overall available power in the building. Although the \ac{EVCS} has access to the current available power in the building, it would be possible to optimize its operation by also feeding it with forecasts of the the near future available power, for example, with predicted values for 5, 10 and 15 minutes ahead. To meet this requirement, a system consisting of \ac{ML} techniques was developed. Several factors, both meteorological as well as energetic, have been used. The implementation was accomplished using three different architectures, Vanilla \ac{RNN}, Encoder-Decoder and \ac{1D CNN}-Encoder-Decoder. For each of the three architectures, two types of \ac{RNN}s were tested, \ac{GRU} and \ac{LSTM}, thus resulting in a total of 6 different models: \ac{GRU} Vanilla, \ac{LSTM} Vanilla, \ac{GRU}-Encoder-Decoder, \ac{LSTM}-Encoder-Decoder, \ac{1D CNN}-\ac{GRU}-Encoder-Decoder and \ac{1D CNN}-\ac{LSTM}-Encoder-Decoder. It was also attempted to introduce \ac{MCD} to the proposed architectures, allowing a probabilistic interpretation of the results obtained, and the computation of confidence intervals of the forecasts carried out.
